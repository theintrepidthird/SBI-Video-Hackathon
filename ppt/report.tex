\documentclass{beamer}
\usetheme{Boadilla}
\usepackage{geometry}                		% See geometry.pdf to learn the layout options. There are lots.
%\geometry{letterpaper}                   		% ... or a4paper or a5paper or ... 
%\geometry{landscape}                		% Activate for rotated page geometry
\usepackage[parfill]{parskip}    			% Activate to begin paragraphs with an empty line rather than an indent
\usepackage{graphicx}				% Use pdf, png, jpg, or eps§ with pdflatex; use eps in DVI mode
								% TeX will automatically convert eps --> pdf in pdflatex		
\usepackage{amssymb}
\usepackage{mathtools}
\usepackage{hyperref}
\usepackage{enumerate}
\usepackage{tikz}

\usetikzlibrary{arrows}

\title{Optimal Hyper-parameterization for HEVC}
\subtitle{State Bank of India}
\author{Shubh Kumar}
\institute{IIT Bombay}
\date{\today}

\begin{document}

\begin{frame}
    \frametitle{Optimal Hyper-parametrization for efficient Video Compression}
    \begin{center}
        \includegraphics[scale=0.02]{im.png}\\
        \textbf{Video Compression Hackathon - SBI} \\

        Powered by Microsoft Corporation Pvt. Ltd.
    \end{center}


\end{frame}



\begin{frame}
    \frametitle{Problem}
    \begin{itemize}
        \item With the advent of Video based Customer Identification, multiple use-cases have emerged for customer onboarding in a secure, paperless, cost-effective and friendly manner.% \item With the advent of Video Based Customer Identification Process (V-CIP), multiple use cases have emerged for customer onboarding & servicing requests of customers in a secure, paperless, cost-effective and friendly manner.
        \item Storage and retrieval of these video files is a challenge especially given the expected tsunami of video files that are expected to be generated on extending more use-cases to our client base of over 45 crore customers.
    \end{itemize}
\end{frame}


\begin{frame}
    \frametitle{Approach}
    \begin{itemize}
        \item Reinforcement Learning Models provide great methods for getting optimal hyperparameters for other methods/models.
        \item These methods have been used heavily in all domains.
        \item I made use of the same to learn optimal parameters for a vast array of these.
    \end{itemize}
\end{frame}




\begin{frame}
    \frametitle{Approach}
    \begin{itemize}
        \item These included (among others!) :  Maximum Reference to L0, Early Skipping etc.
        \item These methods were tabulated into 4 optimal settings in \texttt{vals.txt}
        \item Then, these were made use of for encoding using \texttt{x265}.
    \end{itemize}
\end{frame}



\begin{frame}
    \frametitle{Approach}
    \begin{itemize}
        \item The RL part was done using Feedback Networks' Architecture in TensorFlow.
        \item The Encryption Algorithm used was AES.
        \item We also used \texttt{zlib} for further compression.
    \end{itemize}
\end{frame}


\begin{frame}
    \frametitle{Video used}
    \begin{itemize}
        \item \texttt{phase3.mp4}
        \item Profile : H.264
        \item Dimensions : $1920 \times 1080$
        \item FPS : 30.0
        \item Bit-Rate : 17036 kbps
        \item Size : 127.8 MB
        \item The file may be found \href{https://drive.google.com/file/d/14fxNcPJBfU-HgPigVKemZpu6zYA3YGQZ/view?usp=sharing}{here}.
    \end{itemize}
\end{frame}

\begin{frame}
    \frametitle{Decompressed Video}
    \begin{itemize}
        \item \texttt{phase31.mp4}
        \item Profile (Main) : H.265
        \item Dimensions : $1920 \times 1080$
        \item FPS : 30.0 (as expected)
        \item Bit-Rate : 1381 kbps
        \item Size (compressed) : 9.5 MB
    \end{itemize}

\end{frame}


\begin{frame}
    \frametitle{Hardware Details}
    \begin{itemize}
        \item CPU : Intel(R) Core(TM) i5-1035G1 CPU \@ 1.00GHz

        \item Memory : 8 GB
        \item Memory Clock : 3200 MHz
        \item L1 Cache : 128 kB
        \item L2 Cache : 2 MiB
        \item L3 Cache : 6 MiB
    \end{itemize}
\end{frame}

\begin{frame}
    \frametitle{Why this model?}
    \begin{itemize}
        \item Different Types of applications provide scope for different types of compression shorthands.
        \item Our approach enables us to learn these shorthands
    \end{itemize}
\end{frame}

\begin{frame}
    \frametitle{Functional Requirements}
    \begin{itemize}
        \item \texttt{x265}
        \item \texttt{opencv4}
        \item Python's \texttt{cryptography}
        \item Python's \texttt{zlib}
        \item \texttt{ffmpeg}
    \end{itemize}
\end{frame}

\begin{frame}
    \frametitle{Corresponding Microsoft Tools}
    \begin{itemize}
        \item \texttt{x265} $\equiv$ \texttt{HEVC Video Extensions}
        \item \texttt{opencv4} $\equiv$ Microsoft Media Foundation \texttt{MFIDL}
        \item \texttt{ffmpeg} $\equiv$ \texttt{FFmpegInterop}
    \end{itemize}
\end{frame}


\begin{frame}
    \frametitle{Non-functional Requirements}

    \begin{itemize}
        \item The Video isn't recorded in camcorders which make use of Compressed Sensing.
        \item The Noise is minimal and patterns in the images remain consistent.
        \item The Machine on which the Application is used has $\geq$ 8GB RAM.
        \item This would enable greater compression due to easier Identification of Generic features.
    \end{itemize}
\end{frame}


\begin{frame}
    \frametitle{Our Edge}
    \begin{itemize}
        \item \texttt{HEVC} is known to better than state-of-the-art Deep Learning Methods (\textit{DeepCoder}) by an additional 50 $\%$.
        \item Use-case specific optimization in \textit{HEVC} is generically obtained by making use of hyperparameter optimization.
        \item Feedback functions could also be tinkered to better reflect our Requirements (should they ever change!)
    \end{itemize}
\end{frame}

\begin{frame}
    \frametitle{Github Repository}

    My Actual Github ID is \href{https://github.com/thevaliantthird}{thevaliantthird}

    I've kept the Video Hackathon submission Repository via another profile, \href{https://github.com/theintrepidthird/SBI-Video-Hackathon}{SBI-Video-Hackathon}
\end{frame}

\begin{frame}
    \frametitle{Video Demonstration}

    I have demonstrated usage of my application \href{https://github.com/theintrepidthird/SBI-Video-Hackathon/blob/main/demonstration.mp4}{here}.
\end{frame}

\begin{frame}
    \begin{center}
        Thank You! \\~\\
        Name : Shubh Kumar  \\
        Email : shubh5796@gmail.com \\
        Phone Number : 9470434205 \\
        Github ID (original) : \href{https://github.com/thevaliantthird}{thevaliantthird}
        Website : \href{thevaliantthird.github.io}{thevaliantthird}
        
    \end{center}

\end{frame}

\end{document}